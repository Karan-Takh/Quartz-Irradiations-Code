\documentclass{article}
\usepackage[utf8]{inputenc}
\usepackage{multirow}
\usepackage{graphicx}


\title{Koeth Report}
\author{Karandeep Takhtani, Karen Wiratan, Wil Morrissey, Nancy Chen}
\date{July 2021}

\begin{document}

\maketitle

\section{Introduction}

The objective of this lab was to study spectral information regarding the response of quartz to electron radiation damage. The understanding of this radiation damage can be quantified by determining the location of the spectator neutrons within the material through the use of a CMS Spectator Reaction Plane Detector (SRPD). This detector consists of a matrix of quartz blocks and fiber optics. The SRPD is located inside of a Zero Degree Calorimeter (ZDC) which measures the energy of heavy ion collisions within the sample in order to determine the position of the spectator neutrons.
Cherenkov radiation is a response of the quartz matrix within the SRPD to its interactions with particles which are created when neutrons and protons interact with the ZDC. This radiation response is induced when charged particles (EM showers) interact with the dielectric material and are traveling at velocities greater than the phase velocity of light within the dielectric medium. When the velocity of these particles exceeds a certain threshold, the fibers within the SRPD are illuminated which is a phenomena known as the Cherenkov Effect. The response of the resulting light is proportional to the intensity of the particles that interact with the ZDC. 
This Cherenkov light that radiates from the matrix within the SRPD quantifies the radiation damage response of the quartz. Moreover, the radiation damage is assessed by means of changes in electrical signals which are created when total reflection allows for the light to distribute across the entire fiber and is converted into the electric signal by means of a photomultiplier. Ultimately, this experiment uses CMS technology in order to assess the response of quartz samples to radiation and quantify the radiation damage by means of signals generated by the interactions of light with an SPRD. 

\section{Methods}

A Jenway 6715 UV/Visible scanning spectrophotometer was retrofitted with a single delrin holder to place the 2 cm cylindrical quartz samples for light transmission measurement through the bulk and parallel to the axis of rotation. One preliminary electron beam irradiation was on SRPD-like quartz at 4MeV, similar to the EM shower energies found during CMS operations. This quartz sample, named Sample 6, was cut and polished from 1 cm diameter rod stocks. It was wrapped with foil and analyzed in darkroom settings at all times. Before irradiation, it was heated at ~200°C for 20 minutes as an initial thermal reset.
Nine irradiations were performed and the transmission percentage was measured four times in between each irradiation. Dosimetry with a series of four radiochromic films (swapped out at different intervals) was performed per irradiation to assess whether a consistent dosage was delivered. The irradiation scheme with altering radiochromic films is shown below in Table 1. Each irradiation exposed the quartz at a repetition rate of 150PPS for a duration of 2 minutes. During irradiation, both the sample and radiochromic films were clipped on the side to center the beam on them. The true location of the center of the beam was unknown, so a plastic sheet was used beforehand to determine the location. Electron beam scattering in the air was large compared to the size of the quartz samples, so the electron beam was considered uniform over the samples. A low dose rate was maintained in order to preserve the samples’ temperature below 50°C such that in situ thermal self-annealing was avoided.\par


\begin{center}
    \begin{tabular}{ |c|c|c|c|c|c|c|c|c|c|c|c|c|c|c| }
    \hline
    \multicolumn{1}{|c|}{ } &
    \multicolumn{10}{|c|}{\textbf{Number of Irradiations}}\\
    \hline
     \textbf{Series of Film Tracks} & 1 & 2 & 3 & 4 & 5 & 6 & 7 & 8 & 9 & 10 \\
     \hline
     1 & \multicolumn{4}{|c|}{1A} & \multicolumn{3}{|c|}{1B} & \multicolumn{3}{|c|}{1C} \\
    \hline
    2 & \multicolumn{3}{|c|}{2A} & \multicolumn{3}{|c|}{2B} & \multicolumn{4}{|c|}{2C} \\
    \hline
    3 & \multicolumn{5}{|c|}{3A} & \multicolumn{5}{|c|}{3B} \\
    \hline
    4 & \multicolumn{1}{|c|}{4A} & \multicolumn{9}{|c|}{ } \\
    \hline
    \end{tabular}
    
    \textbf{Table 1.} \\
    Irradiation scheme to assess the accuracy and precision of dosage received. Each irradiation was two minutes long and eight minutes apart from each other to account for the time taken to transfer the sample to the spectrophotometer and back to the LINAC.
\end{center}

\section{Results}
\begin{figure}[h]
\centering
\includegraphics[height=5cm]{graph1.png}
\caption {Optical transmission comparison of sample 4 after ten irradiations were performed. The black, blue, green, red, cyan, and magenta solid lines were respectively the standard and first to fifth irradiations. The black, blue, green, red, cyan, and magenta dashed lines were respectively the sixth to tenth irradiations. There was a clear change in percent transmission as the number of irradiations increased.}
\end{figure}

\begin{figure}[h]
\centering
\includegraphics[height=5cm]{graph2.png}
\caption{ Optical transmission comparison of sample 4 after ten irradiations, comprising 5 scans back-to-back per irradiation, were performed. The black, blue, green, red, cyan, and magenta solid lines were respectively the standard and first to fifth irradiations. The black, blue, green, red, cyan, and magenta dashed lines were respectively the sixth to tenth irradiations. There was a definite change in percent transmission over time.}
\end{figure}

\begin{figure}[h]
\centering
\includegraphics[height=5cm]{graph3.png}
\caption{Optical transmission comparison of sample 4 after one day and five days of rest. The black, blue, and green solid lines were respectively the standard, one day of rest (day 2), and five days of rest (day 6). Scans were conducted five times per hour for eight hours per day. There was no obvious change in percent transmission over an eight-hour period per day.}
\end{figure}

\begin{center}
    \begin{tabular}{ |c|c|c|c|c|c|c|c|c|c|c|c|c|c|c| }
    \hline
    \multicolumn{1}{|c|}{ } &
    \multicolumn{10}{|c|}{\textbf{Number of Irradiations}}\\
    \hline
     \textbf{Series of Film Tracks} & 1 & 2 & 3 & 4 & 5 & 6 & 7 & 8 & 9 & 10 \\
     \hline
     1 & \multicolumn{4}{|c|}{1A - 29.5 kGy} & \multicolumn{3}{|c|}{1B - 28.9 kGy} & \multicolumn{3}{|c|}{1C - 12.6 kGy} \\
    \hline
    2 & \multicolumn{3}{|c|}{2A - 19.5 kGy} & \multicolumn{3}{|c|}{2B - 31.0 kGy} & \multicolumn{4}{|c|}{2C - 12.6 kGy} \\
    \hline
    3 & \multicolumn{5}{|c|}{3A - 36.7 kGy} & \multicolumn{5}{|c|}{3B - 29.0 kGy} \\
    \hline
    4 & \multicolumn{1}{|c|}{4A - 5.2 kGy} & \multicolumn{9}{|c|}{ } \\
    \hline
    \end{tabular}
    
    \textbf{Table 2.} \\
    Irradiation scheme with the accrued dosage received per film track. The total accrued dosage for the three series of film tracks were respectively 192kGy, 198kGy, and 193kGy. The approximate dosage received per irradiation was 20kGy.
   
\end{center}

\section{Discussion}
The PMT sensitive region in the wavelength range of 200nm to 600nm is of particular interest because that is the response region of multi-anode photomultiplier tubes (PMTs) within the SRPD. Figure 1 shows a dramatic reduction in transmission compared to the unirradiated control. It is also noticeable that the wavelength band has stretched after irradiation. In particular, the narrow peak showing 28% transmission at 235nm in the standard black solid line has changed to a broader peak displaying 6% to 8% transmission at 275nm after irradiation. Likewise, the wavelength band of 250nm to 400nm in the standard black solid line has stretched to a very wide band of 310nm to 665nm. 
Figure 2 shows the reduction in transmission over time as five transmission scans were performed in between each irradiation. Each irradiation was two minutes long and eight minutes apart from each other to account for the time taken to transfer the sample to the UV-Vis spectrophotometer, scan the sample, and transfer it back to the LINAC. The percent transmission  per irradiation slightly varied in the PMT sensitive region of 200nm to 600nm whereas it greatly varied in the IR region. This means that the PMTs within the SRPD are sensitive and precise at detecting light. 
Figure 3 shows the variation in transmission over a longer period of time as five transmission scans were performed per hour for eight hours on two different days. The individual scans per day are not shown as there was no dramatic change in percent transmission. However, it is notable that percent transmission did increase from day 2 to day 6, meaning that the quartz sample was indeed recovering over time. In particular, the 275nm to 300nm range showed the biggest increase approximately three percent. 
Looking at Table 2, the dosage received per two minutes was approximately 20 kGy. This dosage will be used in the next experiment which will feature a more detailed irradiation scheme consisting of twenty irradiations. It is assumed that the quartz sample drastically increases upon the first irradiation, so the next plan is to perform ten 0.2 minute irradiations at 5kGy and ten 2 minute irradiations at 20kGy. This will potentially allow a closer observation of the percent transmission over time, especially in the beginning of the irradiation process. 
In a previous experiment, one of the goals was to find a method to anneal (or heal) the quartz of the radiation damage. If full annealing is not obtainable, the alternative is to subject the quartz to an external physical annealing process, such as heat or exposure to UV light, to see if it will return to a known state. An initial thermal mitigation was performed by heating sample 4 to 200°C for one hour before irradiation; however, the data was not retrievable. The next experiment will also feature heating a brand new quartz sample at 200°C for one hour as an initial thermal reset.

\section{onclusion}
\end{document}
